% report_2_01.tex
% Report on task 2.01
% Vladimir Rutsky <altsysrq@gmail.com>
% 21.12.2009

\documentclass[a4paper,12pt]{article}

% Encoding support.
\usepackage{ucs}
\usepackage[utf8x]{inputenc}
\usepackage[T2A]{fontenc}
\usepackage[russian]{babel}

\usepackage{amsmath, amsthm, amssymb}

% Indenting first paragraph.
\usepackage{indentfirst}

\usepackage{listings}
% Source code listings.
\renewcommand{\lstlistingname}{Исходный код}

% Spaces after commas.
\frenchspacing
% Minimal carrying number of characters,
\righthyphenmin=2

% From K.V.Voroncov Latex in samples, 2005.
\textheight=24cm   % text height
\textwidth=16cm    % text width.
\oddsidemargin=0pt % left side indention
\topmargin=-1.5cm  % top side indention.
\parindent=24pt    % paragraph indent
\parskip=0pt       % distance between paragraphs.
\tolerance=2000
%\flushbottom       % page height aligning
%\hoffset=0cm
%\pagestyle{empty}  % without numeration

\title{Отчет по курсу ``Операционные системы'' \\ Задание 2.01}
\author{Владимир Руцкий, 4057/2}

\newcommand{\commandquote}[1]{\textbf{#1}}

\begin{document}

% Title page.
\maketitle
% Content

\section*{Постановка задачи}
\textit{``Напишите программу, вычисляющую суммарный размер всех файлов в указанном
каталоге и его подкаталогах.''}

\section*{Выбранный метод решения}
\begin{enumerate}
 \item Программа принимает на вход список каталогов (файлов),
суммарный размер файлов в которых требуется вычислить.
 \item Для вычисления суммарного размера файлов в каталоге используется рекурсивная процедура,
принимающая на вход текущий обрабатываемый каталог (файл).
 \item Процедура определяет тип обрабатываемого файла с помощью функции \commandquote{stat()} и, 
если это файл, то возвращает его размер,
а если это каталог, то вызывает себя от файлов в каталоге 
(считывание содержимого каталога осуществляется функциями \commandquote{opendir()}, 
\commandquote{readdir()} и \commandquote{closedir()}).
\end{enumerate}

\section*{Исходный код}
\lstset{language=bash, caption=task\_2\_01.c,%
label=source-code, basicstyle=\footnotesize,%
numbers=left, numberstyle=\footnotesize, numbersep=5pt, frame=single, breaklines=true, breakatwhitespace=false,%
inputencoding=utf8x}
\lstinputlisting{data/task_2_01.c}

\end{document}
