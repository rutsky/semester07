% report_4_02.tex
% Report on task 4.02
% Vladimir Rutsky <altsysrq@gmail.com>
% 21.12.2009

\documentclass[a4paper,12pt]{article}

% Encoding support.
\usepackage{ucs}
\usepackage[utf8x]{inputenc}
\usepackage[T2A]{fontenc}
\usepackage[russian]{babel}

\usepackage{amsmath, amsthm, amssymb}

% Indenting first paragraph.
\usepackage{indentfirst}

\usepackage{listings}
% Source code listings.
\renewcommand{\lstlistingname}{Исходный код}

% Spaces after commas.
\frenchspacing
% Minimal carrying number of characters,
\righthyphenmin=2

% From K.V.Voroncov Latex in samples, 2005.
\textheight=24cm   % text height
\textwidth=16cm    % text width.
\oddsidemargin=0pt % left side indention
\topmargin=-1.5cm  % top side indention.
\parindent=24pt    % paragraph indent
\parskip=0pt       % distance between paragraphs.
\tolerance=2000
%\flushbottom       % page height aligning
%\hoffset=0cm
%\pagestyle{empty}  % without numeration

\title{Отчет по курсу ``Операционные системы'' \\ Задание 4.02}
\author{Владимир Руцкий, 4057/2}
\date{24 декабря 2009~г.}

\newcommand{\commandquote}[1]{\textbf{#1}}
\newcommand{\varquote}[1]{\textbf{#1}}

\begin{document}

% Title page.
\maketitle
% Content

\section*{Постановка задачи}
\textit{``Напишите программу, моделирующую решение задачи читателей и писателей в
условиях, когда разрешено одновременное чтение данных, а запись требует
монопольного доступа.''}

\section*{Выбранный метод решения}
Для работы с потоками используются потоки POSIX.

Используется два мутекса:
\begin{itemize}
 \item \varquote{databaseMutex}~--- взаимно исключающий доступ к базе данных.
Им обладает либо один из писателей, который в данный момент пишет в базу, 
либо группа читателей, которые параллельно читают.
 \item \varquote{nReadersMutex}~--- взаимно исключающий доступ к счетчику читателей.
\end{itemize}

При запуске программа создаёт по одному потоку для каждого их читателей и писателей 
(\commandquote{pthread\_create}),
дальше они работают независимо друг от друга, координируя действия только через мутексы.

Работа писателя:
\begin{enumerate}
 \item Захват \varquote{databaseMutex} (\commandquote{pthread\_mutex\_lock}).
 \item Работа с базой данных.
 \item Освобождение \varquote{databaseMutex} (\commandquote{pthread\_mutex\_unlock}).
\end{enumerate}

Работа читателя:
\begin{enumerate}
 \item Захват \varquote{nReadersMutex}.
 \item Если количество читателей равно нулю~--- текущий читатель единственный, 
а значит доступ к базе не захвачен~--- захват \varquote{databaseMutex}.
 \item Увеличение числа читателей.
 \item Освобождение \varquote{nReadersMutex}.
 \item Чтение базы.
 \item Захват \varquote{nReadersMutex}.
 \item Если количество читателей равно одному~--- читатель был последним, 
доступ к базе данных больше не нужен~--- освобождение \varquote{databaseMutex}.
 \item Уменьшение числа читателей.
 \item Освобождение \varquote{nReadersMutex}.
\end{enumerate}

\section*{Исходный код}
\lstset{language=C, caption=task\_4\_02.c,%
label=source-code, basicstyle=\footnotesize,%
numbers=left, numberstyle=\footnotesize, numbersep=5pt, frame=single, breaklines=true, breakatwhitespace=false,%
inputencoding=utf8x}
\lstinputlisting{data/task_4_02.c}

\end{document}
