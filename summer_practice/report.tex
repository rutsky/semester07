% report.tex
% Report on summer 2009 four weeks practice.
% Vladimir Rutsky <altsysrq@gmail.com>
% 08.09.2009

\documentclass[a4paper,10pt,titlepage]{report}

% Encoding support.
\usepackage{ucs}
\usepackage[utf8x]{inputenc}
\usepackage[T2A]{fontenc}
\usepackage[russian]{babel}

\usepackage{amsmath, amsthm, amssymb}

% Spaces after commas.
\frenchspacing
% Minimal carrying number of characters,
\righthyphenmin=2

% From K.V.Voroncov Latex in samples, 2005.
\textheight=24cm   % text height
\textwidth=16cm    % text width.
\oddsidemargin=0pt % left side indention
\topmargin=-1.5cm  % top side indention.
\parindent=24pt    % paragraph indent
\parskip=0pt       % distance between paragraphs.
\tolerance=2000
%\flushbottom       % page height aligning
%\hoffset=0cm
%\pagestyle{empty}  % without numeration

\begin{document}

% Title page.
\begin{titlepage}
\newpage

\begin{center}
% TODO: What is `\\*'?
САНКТ-ПЕТЕРБУРГСКИЙ \\*
ГОСУДАРСТВЕННЫЙ ПОЛИТЕХНИЧЕСКИЙ УНИВЕРСИТЕТ \\*
\hrulefill
\end{center}

\flushright{КАФЕДРА ПРИКЛАДНОЙ МАТЕМАТИКИ}

\vspace{8em}

\begin{center}
\Large Отчет \\ по летней практике
\end{center}

\vspace{8.5em}

\begin{center}
Студента группы 4057/2 \\ Руцкого Владимира
\end{center}

\vspace{\fill}

\end{titlepage}
\pagebreak

% Content

\section*{Постановка задачи}
На летней практике я занимался частным случаем оптимизации триангуляции ландшафта.

Популярной задачей создания трёхмерной реалистичной сцены является построение ландшафта, 
с нанесёнными на него контурами, заполненными текстурами дороги, песка, травы и~т.\,п.

Ландшафт, как и другие объекты трёхмерной сцены, должен быть представлен набором треугольников.
Особенностью ландшафта как объекта сцены в данном случае является то, что он является $\partial 3\mathrm{D}$~---~
это планарный трёхмерный объект.
По сути это плоскость земли с выступами вверх в виде неровностей рельефа.

Ландшафт, сам по себе, представляет собой триангулированную карту высот.

Для внесения контура в триангуляцию ландшафта, для последующего наложения на него текстуры,
в триангуляцию добавляют дополнительные рёбра ограничений\footnote{англ.~constraint} по форме контура.

Часто возникают ситуации, когда различные контура граничат друг с другом.
Например: контур границы района города, по которому обрезаются контура дорог и травы,
смежен своими рёбрами с рёбрами контуров, которые он обрезает.
В таком случае в триангуляции может найтись отрезок ограничения,
содержащий в себе несколько других отрезков ограничений.
Таким образом в триангуляции появляется набор перекрывающих друг-друга рёбер не имеющих общих вершин.

Здесь возникает проблема: из-за погрешности вещественных типов данных в компьютере,
часто возникает ситуация, что хотя перекрывающие друг-друга рёбра были получены как пересечение
каких-то контуров с одним ребром секущего контура, 
точки перекрывающих друг-друга рёбер не будут строго принадлежать одной прямой (или секущему ребру).

Система сцен рассчитана на возможность одновременного использования кусков сцен из разных частей Земного шара.
Поэтому используется двухуровневая система позиционирования объектов: 
позиция \textit{базы} в широте--долготе и позиция XY в локальной системе координат базы.
Для перехода между системами координат баз используются различные 
проекции\footnote{например, проекция Меркатора, проекция Гаусса--Крюгера}, 
в некоторой степени учитывающие форму Земли.

В результате использования дополнительных преобразований систем координат при построении сцен, 
теряется точность 
и, при различных проекциях точек плоскости в систему координат наложенную на геоид и обратно,
точки, лежащие на одной прямой в плоскости, довольно грубо перестают лежать на одной прямой.

Здесь можно определить такое понятие как \textit{T-вершина}\footnote{англ.~T-vertex}~--- 
это вершина ребра, касающаяся внутренней части другого ребра.

Выше была показана возможная причина возниконовения T-вершин.
Существенным минусом Т-вершин является то, 
что в триангуляции из-за них возникают длинные вытянутые треугольники~--- 
``дырки'' триангуляции математически точно полученных рёбер.
Такие треугольники являются как визуально ненужными, 
так и могут вносить артефакты в сцену
(т.\,к.~внутри таких треугольников, вообще говоря, не должно быть никакой текстуры).

Также из-за Т-вершин может нарушаться инвариант принадлежности точки сцены контуру,
что несёт дополнительные артифакты в работе различных алгоритмов.

Моей задачей было выявление причин возникновения таких треугольников и их элиминация.

\section*{Выбранный метод решения}
Для уничтожения T-вершин было решено подразбивать перекрывающие друг-друга рёбра так, 
чтобы Т-вершина входила во все проходящие через неё рёбра 
(и поэтому не являлась T-вершиной в конечной сцене).

Для этого в алгоритмы базовых операций с контурами была внесена возможность ``обратной связи'' в подразбиении: 
при добавления новой вершины в результат операции (пересечения контуров, например), 
вершина также вносилась и в оригинальные контура.

Необходимо было отследить все места в процессе построения сцены, 
где могли образоваться Т-вершины и внести их в лежащие на сцене контура.

В алгоритме генерации сцены были выявлены основные места, 
где происходит сечение какими-либо контурами.
Все подразбитые секущие контуры сохранялись в специальном хранилище.

Далее необходимо было найти контура, смежные с секущими контурами, 
чтобы внести в них T-вершины.
Для этого строилась двухуровневая сетка по всем контурам сцены~---
сетка, клетками которой являются сетки, построенная так, 
что в каждой маленькой клетке двухуровневой сетки в среднем было одинаковое число элементов 
(отрезков контуров).
Чтобы получить контура, смежные к секущем контурам, достаточно было растеризовать секущие 
контура (с их epsilon-окрестностью) по построенной сетке и проверить на смежность 
найденные в клетках отрезки.

Теперь, имея информацию о контурах, смежных с секущими контурами, 
необходимо было внести T-вершины в контура.

Элементарной подзадачей полученной задачи было подразбиение контура вершинами другого контура.
Для этого по контору, в которой ``проталкивались'' вершины строилась одноуровневая сетка, 
и по ней производилось растеризация каждой точки второго контура. 
Если растеризированная вершина оказывалась в epsilon-окрестности от ребра подразбиваемого контура, 
она вносилась в контур.

\section*{Результаты}
На сцене из порядка 70 тысяч контуров с около 6000 секущими контурами было внесено порядка 5000 дополнительных вершин.

В результате был элиминирован класс артифактов и незначительно уменьшено число полигонов в сцене.

% Quick and dirty.
\begin{thebibliography}{9}
\bibitem{prepara-sheimos}
  Ф.~Препарата, М.~Шеймос
  \emph{Вычислительная геометрия: Введение}.
  Мир,
  1989.
\bibitem{kormen}
  Т.\,Х.~Кормен, Ч.\,И.~Лейзерсон
  \emph{Алгоритмы: построение и анализ.}
  Вильямс,
  2-е издание,
  2008.
\end{thebibliography}

\end{document}
