% report.tex
% Report on summer 2009 four weeks practice.
% Vladimir Rutsky <altsysrq@gmail.com>
% 08.09.2009

\documentclass[a4paper,10pt,titlepage]{report}

% Encoding support.
\usepackage{ucs}
\usepackage[utf8x]{inputenc}
\usepackage[T2A]{fontenc}
\usepackage[russian]{babel}

\usepackage{amsmath, amsthm, amssymb}

% Spaces after commas.
\frenchspacing
% Minimal carrying number of characters,
\righthyphenmin=2

% From K.V.Voroncov Latex in samples, 2005.
\textheight=24cm   % text height
\textwidth=16cm    % text width.
\oddsidemargin=0pt % left side indention
\topmargin=-1.5cm  % top side indention.
\parindent=24pt    % paragraph indent
\parskip=0pt       % distance between paragraphs.
\tolerance=2000
%\flushbottom       % page height aligning
%\hoffset=0cm
%\pagestyle{empty}  % without numeration

\title{Отчёт по летней практике}
\author{студента группы 4057/2, Руцкого Владимира}
\date{} % cheat ;)

\begin{document}

% Title page.
\begin{titlepage}
\newpage

\begin{center}
% TODO: What is `\\*'?
САНКТ-ПЕТЕРБУРГСКИЙ \\*
ГОСУДАРСТВЕННЫЙ ПОЛИТЕХНИЧЕСКИЙ УНИВЕРСИТЕТ \\*
\hrulefill
\end{center}

\flushright{КАФЕДРА ПРИКЛАДНАЯ МАТЕМАТИКА}

\vspace{8em}

\begin{center}
\Large Отчет \\ по летней практике
\end{center}

\vspace{8.5em}

\begin{center}
Студента группы 4057/2 \\ Владимира Руцкого
\end{center}

\vspace{\fill}

\end{titlepage}
\pagebreak

% Content

\section*{Постановка задачи}
На летней практике я занимался частным случаем оптимизации триангуляции набора геометрических объектов.

Популярной задачей создания трёхмерной реалистичной сцены является построение ландшафта, 
с нанесёнными на него контурами, заполенными текстурами дороги, песка, травы и~т.\,п.

Ландшафт, как и другие объекты трёхмерной сцены, должен быть представлен набором треугольников.
Особенностью ландшафта как объекта сцены в данном случае является то, что он является $\partial 3D $~---~
это планарный трёхмерный объект.
По сути это плоскость земли с выступами вверх в виде неровностей рельефа.

Ландшафт сам по себе представляет собой триангулированную карту высот.

Для внесения контура в триангуляцию ландшафта для последующего наложения на него текстуры в триангуляцию
добавляют дополнительные рёбра ограничений\footnote{англ.~constraint} по форме контура.

Часто возникают ситуации, когда различные контура граничат друг с другом.
Например: контур границы района города, по которому обрезаются контура дорог и травы,
смежен своими рёбрами с рёбрами контуров, которые он обрезает.
В таком случае в триангуляции может найтись отрезок ограничения,
содержащий в себе несколько других отрезков ограничений.
Таким образом в триангуляции появляется набор перекрывающих друг-друга рёбер не имеющих общих вершин.

Здесь возникает проблема: из-за погрешности вещественных типов данных в компьютере,
часто возникает ситуация, что хотя перекрывающие друг-друга рёбра были получены как пересечение
каких-то контуров с одним ребром секущего контура, 
точки перекрывающих друг-друга рёбер не будут строго принадлежать одной прямой (или секущему ребру).

Используемая система сцен рассчитана на возможность одновременного использования кусков сцен из разных частей Земного шара.
Поэтому используется двухуровневая система позиционирования объектов: 
позиция ``базы'' в широте--долготе и позиция XY в локальной системе координат.
Для перехода между системами координат ``баз'' используются различные проекции\footnote{например, проекция Меркатора, проекция Гаусса Крюгера}, в некоторой степени учитывающие форму Земли.

В результате использования дополнительных преобразований систем координат при построении сцен, 
теряется точность 
и, при различных проекциях точек плоскости в систему координат наложенную на геоид и обратно,
точки, лежащие на одной прямой в плоскости, довольно грубо перестают лежать на одной прямой.

Здесь можно определить такое понятие как \textit{T-вершина}\footnote{англ.~T-vertex}~--- 
это вершина ребра, касающаяся внутренней части другого ребра.

Выше была показана возможная причина возниконовения T-вершин.
Существенным минусом Т-вершин является то, 
что в триангуляции из-за них возникают длинные вытянутые треугольники~--- 
``дырки'' триангуляции математически точно полученных рёбер.
Такие треугольники являются как визуально ненужными, 
так и могут вносить артефакты в сцену
(т.\,к.~между внутри таких треугольников, вообще говоря, не должно быть никакой текстуры).

Моей задачей было выявление причин возникновения таких треугольников и их элиминация.

\section*{Выбранный метод решения}
\section*{Результаты}
\section*{Список литературы}

\end{document}
